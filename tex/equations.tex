This document describes finite element formulation of the numerical algorithm
for modeling endocytosis given in \cite{lowengrub2016numerical}.

Some notational preliminaries: (i) $(u, v)$ denotes integral of inner product
of $u$ and $v$ over domain $\Omega$, (ii) if $u$ is a trial function from some
function space then a test function is denoted by $\hat{u}$, (iii) $u:v$ is
the inner product of matrices $u, v$. The inner product of vectors $u, v$ is
$u\cdot v$.

Following the authors \cite{lowengrub2016numerical}, see \S 3 therein,
the formulation is presented in three steps. The problem is posed in
domain $\Omega$. Parameters $\epsilon$, $\zeta$, $\gamma$ and time step
$\tau$ are given in addition to functions $b$ and $H_0$ which map scalar
fields ($c$) to numbers. Finally, density $\rho$ and viscosity $\nu$ are
given by $\rho=\rho_1(1+\phi)/2 + \rho_2(1-\phi)/2$ (and analogically
for $\nu$), that is, by linear interpolation of $\phi$ which takes value
$1$ in fluid one having density $\rho_1$ and value $-1$ in the second
fluid whose density is $\rho_2$.

\section*{Variational formulation of step 1}
Tensor valued function $T$ is defined as $T(u, p, \nu)=-pI + 2\nu D(u)$ with
$D(u)$ a symmetric part of the velocity gradient,
$D(u)=\text{sym}(\nabla u) = (\nabla u + \transp{\nabla} u)/2$.
Note that if $A$ is an arbitraty tensor then $D(u):A=D(u):\text{sym}(A)$,
in particular $D(u):\nabla u=D(u):D(u)$. To simplity the notation let
$\mathcal{F}$ be defined as
%
\[
 \mathcal{F}(c, f, \phi) = \frac{3}{4\sqrt{2}}\left[
 \frac{1}{2\epsilon}\dual{b}(c)f^2 -
 \frac{\sqrt{2}}{\epsilon}b(c)f(\phi^2-1)\dual{H_0}(c)
 \right].
\]
%
As in the paper $P(\phi)$ is the tangential projector $P(\phi)=I-n\otimes n$
with $n$ the surface normal $n=\nabla \phi/\semi{\nabla{\phi}}$.

With this notation the strong form of the step is a system 6 PDEs for
6 unknown quantities $\Next{u}$, $\Next{p}$, $\Next{\lambda}$,
$\Next{\phi}$, $\Next{g}$, $\Next{f}$:
%

\begin{equation}\label{eq:strong1}
\begin{aligned}
\rho^n(\frac{u^{n+1}-u^n}{\tau} + u^n\cdot \nabla u^{n+1}) - \textcolor{blue}{\nabla\cdot T(u^{n+1}, p^{n+1}, \nu^n)}
\textcolor{blue}{-\nabla\cdot(\semi{\nabla \phi^n}P(\phi^n)\lambda^{n+1})} +
g^{n+1}\nabla \phi^n =\\ \mathcal{F}(c^n, f^n, \phi^n)\nabla c^n\\
%
\nabla\cdot \Next{u} = 0\\
%
\textcolor{blue}{\zeta\epsilon^2\nabla\cdot\left[(\phi^n)^2\nabla\lambda^{n+1}\right]}
+\semi{\nabla \phi^n}P(\phi^n):\nabla u^{n+1} = \tau^{-1}\semi{\nabla \phi^n}\frac{s^n-1}{s^n}\\
%
\frac{\phi^{n+1}-\phi^n}{\tau} + u^{n+1}\cdot\nabla\phi^n + \gamma g^{n+1} = 0\\
%
g^{n+1}-\frac{3}{4\sqrt{2}}\left[
\Delta(b(c^n)f^{n+1}) - \epsilon^{-2}b(c^n)f^{n+1}\left\{
                      3(\phi^{n+1})^2 - 1 + 2\sqrt{2}\phi^{n+1}\epsilon H_0(c^n)
\right\}
\right]=0\\
%
f^{n+1} - \epsilon\Delta\phi^{n+1}+\epsilon^{-1}\left[
(\phi^{n+1})^2-1
\right]\left[
\phi^{n+1}+\sqrt{2}\epsilon H_0(c^n)
\right]=0\\
%
\end{aligned}
\end{equation}
%
The system \eqref{eq:strong1} is considered with boundary conditions
$u=0$ or $T\cdot m=0$ on $\partial\Omega$, $\lambda=0$ on $\partial \Omega$
and $m\cdot\nabla \phi=0$, $\phi(x, t)=\phi(x, 0)$ on $x\in\partial\Omega$
with $m$ the outer normal of the boundary (of $\Omega$).

Instead of the nonlinear problem \eqref{eq:strong1} we will solve the
linearized version obtained by approximation $f(v)\approx f(w) + df(w)(v-w)$.
The nonlinearities in \eqref{eq:strong1} then become
%
\begin{itemize}
\item $((\phi^{n+1})^2-1)\phi^{n+1}\to ((\phi^n)^2-1)\phi^n+(3(\phi^n)^2-1)(\phi^{n+1}-\phi^n)$
%
\item $(\phi^{n+1})^2-1\to (\phi^n)^2-1 + 2\phi^n(\phi^{n+1}-\phi^n)$
%
\item $f^{n+1}(\phi^{n+1})^2\to f^n(\phi^n)^2+(\phi^n)^2(f^{n+1}-f^n)+f^n2\phi^n(\phi^{n+1}-\phi^n)$
%
\item $f^{n+1}\phi^{n+1}=f^n\phi^n + \phi^n(f^{n+1}-f^{n}) + f^n(\phi^{n+1}-\phi^n)$
%
\end{itemize}
%

\subsection*{Individual equations}
Let $V$, $Q$ be respectively the solution space for velocity and pressure, i.e.
$u^{n+1}, \hat{u}\in V$ and $p, \hat{p}\in Q$. Taking an inner of the first two equations
in \eqref{eq:strong1} with $\hat{u}$ and $\hat{p}$ respectively, integrating over
$\Omega$ and applying divergence theorem on the blue terms
%
\[
(\nabla\cdot T, \hat{u}) = \int_{\partial\Omega} \hat{u} \cdot T \cdot m - (T, \nabla{\hat{u}})
\]
and
\[
-(\nabla\cdot(\semi{\nabla \phi^n}P(\phi^n)\lambda^{n+1}), \hat{u})
=
-\int_{\partial\Omega} \semi{\nabla \phi^n}\lambda^{n+1} \hat{u}\cdot P(\phi^n)\cdot m
+
(\semi{\nabla \phi^n}\lambda^{n+1} P(\phi^n), \nabla\hat{u})
\]
%
while realizing that the boundary terms vanish due to boundary conditions
yields
%
\begin{equation}\label{eq:11}
\begin{aligned}
(\frac{\rho^n}{\tau}u^{n+1}+u^n\cdot\nabla u^{n+1}, \hat{u})+
(T(u^{n+1}, p^{n+1}, \nu^{n}), \nabla u) + (\lambda^{n+1}\semi{\nabla \phi^n} P(\phi^n), \nabla \hat{u})
+(g^{n+1}\nabla \phi^n, \hat{u}) =\\ (\mathcal{F}(c^n, f^n, \phi^n)\nabla c^n, \hat{u})+
(\frac{\rho^n}{\tau}u^{n}, \hat{u})\\
%
(\nabla\cdot u^{n+1}, \hat{p}) = 0.
\end{aligned}
\end{equation}
%

Leting $\Lambda$ be the solution space of $\lambda^{n+1}$ one obtains from
the third equation in \eqref{eq:strong1} (the blue term is treated by the
divergence theorem using that the test functions satisfy $\hat{\lambda}=0$
on $\partial\Omega$)
%
\begin{equation}\label{eq:12}
(\semi{\nabla \phi^{n}}P(\phi^n):\nabla u^{n+1}, \hat{\lambda})
-(\zeta\epsilon^2 (\phi^n)^2 \nabla\lambda^{n+1}, \nabla\hat{\lambda})
=(\tau^{-1}\semi{\nabla \phi^n}\frac{s^n-1}{s^n}, \hat{\lambda})
\end{equation}
%

With $\hat{\phi}\in \Phi$ the fourth equation becomes
%
\begin{equation}\label{eq:13}
(u^{n+1}\cdot \nabla\phi^n, \hat{\phi})+
(\tau^{-1}\phi^{n+1}, \hat{\phi})+
\gamma(g^{n+1}, \hat{\phi}) = (\tau^{-1}\phi^{n}, \hat{\phi})
.
\end{equation}
%

The final two equations in \eqref{eq:strong1} after linearization read
\[
\begin{split}
g^{n+1}&-\textcolor{red}{\frac{3}{4\sqrt{2}}\Delta(b(c^n) f^{n+1})}-
\frac{3\epsilon^{-2}}{4\sqrt{2}}b(c^n) f^{n+1}+
\frac{9\epsilon^{-2}}{4\sqrt{2}}b(c^n) (\phi^n)^2 f^{n+1}+
\frac{9\epsilon^{-2}}{2\sqrt{2}}b(c^n) f^n \phi^n \phi^{n+1}\\
%
&\frac{3\epsilon^{-1}}{2}b(c^n)H_0(c^n)f^n \phi^{n+1}+
 \frac{3\epsilon^{-1}}{2}b(c^n)H_0(c^n)f^{n+1} \phi^{n}
 =\\
&\frac{9\epsilon^{-2}}{2\sqrt{2}}b(c^n) f^n (\phi^n)^2 +
 \frac{3\epsilon^{-1}}{2}b(c^n)H_0(c^n) f^{n} \phi^{n}
\end{split}
\]
and
\[%
\begin{split}
f^{n+1}-\textcolor{blue}{\epsilon \Delta \phi^{n+1}}
+3\epsilon^{-1}(
(\phi^n)^2 &-1
)\phi^{n+1}
+2\sqrt{2}H_0(c^n)\phi^n \phi^{n+1}=\\
%%%
&2\epsilon^{-1}\phi^n(
(\phi^n)^2-1) +
\sqrt{2}H_0(c^n)(
(\phi^n)^2+1
)
\end{split}.
\]
To obtain their weak form the color terms are intergrated by parts. Note that
for the first equation we assume that the surface term vanishes. In the
second one this is true since we have $m \cdot\nabla \phi^{n+1}=0$ on the boundary.
Letting $G$ and $F$ be the test spaces for the equations become
%
\begin{equation}\label{eq:14}
\begin{split}
&(g^{n+1}, \hat{g})
+\frac{3}{4\sqrt{2}}(\nabla (b(c^n) f^{n+1}), \nabla \hat{g}) -
\frac{3\epsilon^{-2}}{4\sqrt{2}}(b(c^n) f^{n+1}, \hat{g})+
\frac{9\epsilon^{-2}}{4\sqrt{2}}(b(c^n) (\phi^n)^2 f^{n+1}, \hat{g})\\
+&\frac{9\epsilon^{-2}}{2\sqrt{2}}(b(c^n) f^n \phi^n \phi^{n+1}, \hat{g})
%
+
 \frac{3\epsilon^{-1}}{2} (b(c^n)H_0(c^n)f^n \phi^{n+1}, \hat{g})+
 \frac{3\epsilon^{-1}}{2} (b(c^n)H_0(c^n)f^{n+1} \phi^{n}, \hat{g})
 =\\
&\frac{9\epsilon^{-2}}{2\sqrt{2}} (b(c^n) f^n (\phi^n)^2, \hat{g}) +
 \frac{3\epsilon^{-1}}{2} (b(c^n)H_0(c^n) f^{n} \phi^{n}, \hat{g})
\end{split}
\end{equation}
%
and
\begin{equation}\label{eq:15}
\begin{split}
&(f^{n+1}, \hat{f})
+(\epsilon \nabla \phi^{n+1}, \nabla\hat{f})
+(3\epsilon^{-1}((\phi^n)^2 -1)\phi^{n+1}, \hat{f})
+(2\sqrt{2}H_0(c^n)\phi^n \phi^{n+1}, \hat{f})=\\
%%%
&(2\epsilon^{-1}\phi^n((\phi^n)^2-1), \hat{f})
+
(\sqrt{2}H_0(c^n)((\phi^n)^2+1), \hat{f}).
\end{split}
\end{equation}

The weak form of \eqref{eq:strong1} finally reads: Find $w\in W$, where
$w=(\Next{u}, \Next{p}, \Next{\lambda}, \Next{\phi}, \Next{g}, \Next{f})$
and $W=V\times Q\times \Lambda \times \Phi \times G \times F$ such that
equations \eqref{eq:11}--\eqref{eq:15} hold for arbitrary $\hat{w}\in W$.
